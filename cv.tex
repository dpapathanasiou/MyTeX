% Curriculum Vitae/Resume Template

% This template is based on the one created by
% Jason Blevins <jrblevin@sdf.lonestar.org>
% http://jblevins.org/projects/cv-template/

\documentclass[letterpaper]{article}

\usepackage{hyperref}
\usepackage{geometry}

% fonts: the mathpazo package provides a Palatino font
\usepackage[T1]{fontenc}
\usepackage[sc,osf]{mathpazo}

% or, to use specific fonts on the system
% (source: http://stackoverflow.com/a/1840608)
%\usepackage{fontspec,xunicode}
%\defaultfontfeatures{Mapping=tex-text,Scale=MatchLowercase}
% /usr/bin/fc-list to get list of available fonts (linux)
% cat /Library/Fonts/fonts.list | grep "\.[(dfont|otf|ttc|ttf)]" (mac osx)
%\setmainfont[Scale=1.0]{Liberation Sans Narrow}
%\setmonofont{Liberation Mono}

% Set your name here
\def\name{Denis Papathanasiou}

% The following metadata will show up in the PDF properties
\hypersetup{
  colorlinks = true,
  urlcolor = blue,
  pdfauthor = {\name},
  pdfkeywords = {},
  pdftitle = {\name Curriculum Vitae},
  pdfsubject = {Curriculum Vitae},
  pdfpagemode = UseNone
}

% Paper settings: 8.5x11
\geometry{
  body={6.5in, 9.0in},
  left=1.0in,
  top=1.0in
}

% Customize page headers
\pagestyle{myheadings}
\markright{\name}
\thispagestyle{empty}

% Custom section fonts
\usepackage{sectsty}
\sectionfont{\rmfamily\mdseries\Large}
\subsectionfont{\rmfamily\mdseries\itshape\large}

% Don't indent paragraphs.
\setlength\parindent{0em}

% Make lists without bullets
% and compact line spacing
\renewenvironment{itemize}{
  \begin{list}{}{
    \setlength{\leftmargin}{1.5em}
    \setlength{\itemsep}{1pt}
    \setlength{\parskip}{0pt}
    \setlength{\parsep}{0pt}
  }
}{
  \end{list}
}

% Actual content starts here
\begin{document}

% Place name at top left
{\LARGE \name}

% alternatively, print name centered and bold:
%\centerline{\huge \bf \name}


% use \input to import separate tex files
% as needed/desired in the overall document
% but place them in a separate folder, to
% avoid any problems with the Makefile

% contact information
% this is the cv contact information: address, email, and phone
% note the use of \href, which creates a clickable hyperlink in
% the resulting pdf file

\small

55 Main Street, Anytown, New York 10019 \hspace{0.25em}
\href{mailto:denis@example.org}{denis@example.org} \hspace{0.25em}
(212) 555-1212

\normalsize


% make each section an \input{} from a
% separate tex file in tex-inputs, e.g.:

%\section*{Work Experience}
%\input{tex-inputs/cv-experience}

%\section*{Education}
%\input{tex-inputs/cv-education}

% (etc.)

\end{document}
